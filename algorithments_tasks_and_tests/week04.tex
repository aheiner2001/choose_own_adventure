
\documentclass[12pt]{amsart}
\usepackage{geometry} % see geometry.pdf on how to lay out the page. There's lots.
\geometry{a4paper} % or letter or a5paper or ... etc
\usepackage[T1]{fontenc}
\usepackage[latin9]{inputenc}
\usepackage{amsmath}
\usepackage{amsaddr}
\usepackage{hyperref}
\usepackage{dirtytalk}
\usepackage{float}
\usepackage{listings}
\usepackage{color}
\usepackage{tikz}


 
\definecolor{codegreen}{rgb}{0,0.6,0}
\definecolor{codegray}{rgb}{0.5,0.5,0.5}
\definecolor{stringcolor}{rgb}{0.7,0.23,0.36}
\definecolor{backcolour}{rgb}{0.95,0.95,0.92}
\definecolor{keycolor}{rgb}{0.007,0.01,1.0}
\definecolor{itemcolor}{rgb}{0.01,0.0,0.49}
 
\lstdefinestyle{mystyle}{
    %backgroundcolor=\color{backcolour},   
    commentstyle=\color{codegreen},
    keywordstyle=\color{keycolor},
    numberstyle=\tiny\color{codegray},
    stringstyle=\color{stringcolor},
    basicstyle=\footnotesize,
    breakatwhitespace=false,         
    breaklines=true,                 
    captionpos=b,                    
    keepspaces=true,                 
    numbers=left,                    
    numbersep=5pt,                  
    showspaces=false,                
    showstringspaces=false,
    showtabs=false,                  
    tabsize=2
}
 
\lstset{style=mystyle}

\lstdefinelanguage{Swift}{
  keywords={associatedtype, class, deinit, enum, extension, func, import, init, inout, internal, let, operator, private, protocol, public, static, struct, subscript, typealias, var, break, case, continue, default, defer, do, else, fallthrough, for, guard, if, in, repeat, return, switch, where, while, as, catch, dynamicType, false, is, nil, rethrows, super, self, Self, throw, throws, true, try, associativity, convenience, dynamic, didSet, final, get, infix, indirect, lazy, left, mutating, none, nonmutating, optional, override, postfix, precedence, prefix, Protocol, required, right, set, Type, unowned, weak, willSet},
  ndkeywords={class, export, boolean, throw, implements, import, this},
  sensitive=false,
  comment=[l]{//},
  morecomment=[s]{/*}{*/},
  morestring=[b]',
  morestring=[b]"
}

\lstset{emph={Int,count,abs,repeating,Array}, emphstyle=\color{itemcolor}}


\title{Week 04}

\date{\today}

\lstset{style=mystyle}

%%% BEGIN DOCUMENT
\begin{document}
\maketitle


Name: 
\\

Collaborators (if any): 

\section{Tasks}

\subsection{Topological Sort}

Apply the DFS-based algorithm to solve the topological sorting problem for the following digraphs:
\\

(a) 

\usetikzlibrary{positioning}
\tikzset{
     every path/.append style = {
        arrows = ->
    }, 
    hidden/.style = {
    	draw= black,
        shape = circle,
        inner sep = 3 pt, 
            outer sep = 3 pt
    }
}


\tikz{
    \node[hidden] (a) {$A$};
    \node[hidden] (b) [right = of a] {$B$};
    \node (c)[hidden] [below left = of a] {$C$};
    \node (d)[hidden] [below left = of b] {$D$};
    \node (e)[hidden] [below right = of b] {$E$};
    \node (f)[hidden] [below right = of c] {$F$};
    \node (g)[hidden] [below right = of d] {$G$};
    \path (a) edge (c);
    \path (a) edge (b);
    \path (b) edge (g);
    \path (b) edge (e);
    \path (c) edge (f);
    \path (d) edge (a);
    \path (d) edge (b);
     \path (d) edge (c);
    \path (d) edge (f);
    \path (d) edge (g);
    \path (g) edge (e);
    \path (g) edge (f);
}


(b) 
\\

\usetikzlibrary{positioning}
\tikzset{
     every path/.append style = {
        arrows = ->
    }, 
    hidden/.style = {
    	draw= black,
        shape = circle,
        inner sep = 3 pt, 
            outer sep = 3 pt
    }
}

\tikz{
    \node[hidden] (a) {$A$};
    \node[hidden] (b) [right = of a] {$B$};
    \node (c)[hidden] [right = of b] {$C$};
    \node (d)[hidden] [right= of c] {$D$};
    \node (e)[hidden] [below right = of a] {$E$};
    \node (f)[hidden] [below right = of b] {$F$};
    \node (g)[hidden] [below right = of c] {$G$};
    \path (a) edge (b);
    \path (b) edge (c);
    \path (c) edge (d);
    \path (d) edge (g);
    \path (e) edge (a);
    \path (f) edge (e);
    \path (f) edge (b);
    \path (f) edge (c);
    \path (f) edge (g);
    \path (g) edge [bend left] (e);
}

\subsection{Topological Sort Proof}

Prove that the topological sorting problem has a solution if and only if it is a DAG. For a graph with n vertices, what is the largest number of distinct solutions the topological sorting problem can have? 



\section{Exercises/Problems}

\subsection{Rexburg Power} 
The electrical network in Rexburg contains n power plants and $n^2$ buildings, pairs of which may be connected to each other through bidirectional wires. There are two options for power to each building. First, by having a wire straight to the power plant. Second, by having a wire to a different building which is recursively powered. Note: no building can be powered by multiple plants. Rexburg plans to install a generator in one power plant to act as an emergency backup. This backup generator would provide power to all buildings powered by that power plant were it to fail.  Given a list, W, of all the wires in the network, describe an $O(n^4)$-time algorithm to determine the power plant where Rexburg should install the generator that would provide backup power to the most buildings upon plant failure. 
\\

\end{document}
